\documentclass[12pt,a4paper]{article}
\usepackage[utf8]{inputenc}
\usepackage[T1]{fontenc}
\usepackage{dsfont} 
\usepackage[polish]{babel}
\usepackage{amsmath}
\usepackage{graphicx}
\usepackage[top=1in, bottom=1.5in, left=1.25in, right=1.25in]{geometry}

\usepackage{subfig}
\usepackage{multirow}
\usepackage{multicol}
\graphicspath{{Imagens/}}
\usepackage{xcolor,colortbl}
\usepackage{float}

\newcommand \comment[1]{\textbf{\textcolor{red}{#1}}}

%\usepackage{float}
\usepackage{fancyhdr} % Required for custom headers
\usepackage{lastpage} % Required to determine the last page for the footer
\usepackage{extramarks} % Required for headers and footers
\usepackage{indentfirst}
\usepackage{placeins}
\usepackage{scalefnt}
\usepackage{xcolor,listings}
\usepackage{textcomp}
\usepackage{color}
\usepackage{verbatim}
\usepackage{framed}

\definecolor{codegreen}{rgb}{0,0.6,0}
\definecolor{codegray}{rgb}{0.5,0.5,0.5}
\definecolor{codepurple}{HTML}{C42043}
\definecolor{backcolour}{HTML}{F2F2F2}
\definecolor{bookColor}{cmyk}{0,0,0,0.90}  
\color{bookColor}

\lstset{upquote=true}

\lstdefinestyle{mystyle}{
	backgroundcolor=\color{backcolour},   
	commentstyle=\color{codegreen},
	keywordstyle=\color{codepurple},
	numberstyle=\numberstyle,
	stringstyle=\color{codepurple},
	basicstyle=\footnotesize\ttfamily,
	breakatwhitespace=false,
	breaklines=true,
	captionpos=b,
	keepspaces=true,
	numbers=left,
	numbersep=10pt,
	showspaces=false,
	showstringspaces=false,
	showtabs=false,
}
\lstset{style=mystyle}

\newcommand\numberstyle[1]{%
	\footnotesize
	\color{codegray}%
	\ttfamily
	\ifnum#1<10 0\fi#1 |%
}

\definecolor{shadecolor}{HTML}{F2F2F2}

\newenvironment{sqltable}%
{\snugshade\verbatim}%
{\endverbatim\endsnugshade}

% Margins
\addtolength{\footskip}{0cm}
\addtolength{\textwidth}{1.4cm}
\addtolength{\oddsidemargin}{-.7cm}

\addtolength{\textheight}{1.6cm}
%\addtolength{\topmargin}{-2cm}

% paragrafo
\addtolength{\parskip}{.2cm}

% Set up the header and footer
\pagestyle{fancy}
\rhead{\hmwkAuthorName} % Top left header
\lhead{\hmwkClass: \hmwkTitle} % Top center header
\rhead{\firstxmark} % Top right header
\lfoot{Alicja Chmura} % Bottom left footer
\cfoot{} % Bottom center footer
\rfoot{} % Bottom right footer
\renewcommand{\headrulewidth}{1pt}
\renewcommand{\footrulewidth}{1pt}

    
\newcommand{\hmwkTitle}{E-commerce Unicorn Apparel w Kubernetes} % Tytuł projektu
\newcommand{\hmwkDueDate}{\today} % Data 
\newcommand{\hmwkClass}{Technologie chmurowe} % Nazwa przedmiotu
\newcommand{\hmwkAuthorName}{Alicja Chmura} % Imię i nazwisko

% trabalho 
\begin{document}
% capa
\begin{titlepage}
    \vfill
	\begin{center}
	\hspace*{-1cm}
	\vspace*{0.5cm}
    \includegraphics[scale=0.55]{imagens/loga.png}\\
	\textbf{Uniwersytet Gdański \\ [0.05cm]Wydział Matematyki, Fizyki i Informatyki \\ [0.05cm] Instytut Informatyki}

	\vspace{0.6cm}
	\vspace{4cm}
	{\huge \textbf{\hmwkTitle}}\vspace{8mm}
	
	{\large \textbf{\hmwkAuthorName}}\\[3cm]
	
		\hspace{.45\textwidth} %posiciona a minipage
	   \begin{minipage}{.5\textwidth}
	   Projekt z przedmiotu technologie chmurowe na kierunku informatyka profil praktyczny na Uniwersytecie Gdańskim.\\[0.1cm]
	  \end{minipage}
	  \vfill
	%\vspace{2cm}
	
	\textbf{Gdańsk}
	
	\textbf{\hmwkDueDate}
	\end{center}
	
\end{titlepage}

\newpage
\setcounter{secnumdepth}{5}
\tableofcontents
\newpage

\section{Opis projektu}
\label{sec:Project}

Projekt powstał ze względu na duże zainteresowanie platformą Unicorn Apparel. Celem tego projektu jest zapewnienie solidnej i skalowalnej infrastruktury dla aplikacji e-commerce przy użyciu Kubernetes. Architektura składa się z aplikacji klienckiej napisanej w React Vite, API serwera w ExpressJS oraz bazy danych grafowej Neo4j.
\begin{figure}[htbp]
    \centering
    \includegraphics[width=1\textwidth]{app.png}
    \caption{Aplikacja uruchomiona w klastrze Kubernetes}
\end{figure}

\subsection{Opis architektury - 8 pkt}
\label{sec:introduction}
Aplikacja jest budowana na klastrze Kubernetes, wykorzystując orkiestrację kontenerów do zapewnienia wysokiej dostępności, skalowalności i łatwości zarządzania. Główne komponenty to:
\begin{enumerate}
    \item Serwis Kliencki: Aplikacja frontendowa dostarczana na porcie 3000.
    \item API Serwera: Backendowy serwis komunikujący się z bazą danych Neo4j i obsługujący logikę biznesową.
    \item Baza Danych Neo4j: Baza grafowa używana do przechowywania i zapytywania danych.
\end{enumerate}
Kubernetes zarządza tymi komponentami za pomocą Deploymentów (klient i serwer), StatefulSeta (baza Neo4j), Serwisów, ConfigMaps, Secrets, PersistentVolume i PersistentVolumeClaima oraz Ingressa, zapewniając płynną komunikację i skalowalność. Aplikacja dostepna z zewnatrz przez Ingress Nginx wysyła˙żądania do API backendu uruchomionego w klastrze, a aplikacja backendowa łączy sie z bazą danych i pobiera lub modyfikuje jej dane. Dodatkowo wykorzystane zostały moduły metryki oraz skalowania HorizontalPodAutoscaler (więcej podów zostaje deployowanych w odpowiedzi na zwiększony traffic) umożliwiajace sprawniejsze zarzadzanie klastrem.



\subsection{Opis infrastruktury - 6 pkt}
\label{sec:Users}
Aplikacja korzysta z zbudowanych wcześniej i udostępnionych na Docker Hub obrazów. Infrastruktura obejmuje klaster Kubernetes, który może działać na dowolnym dostawcy chmury, takim jak AWS, GCP lub Azure. Klaster zawiera:
\begin{itemize}
    \item Sieć: Serwisy są wystawiane za pomocą Deploymentów i Ingressa, umożliwiając zarówno dostęp wewnętrzny, jak i zewnętrzny.
    \item Magazynowanie: Trwałe przechowywanie zarządzane jest za pomocą PersistentVolume i PersistentVolumeClaima, zapewniając niezawodność i dostępność danych.
    \item Zarządzanie zasobami: Limity i żądania zasobów są ustawiane w celu optymalizacji wykorzystania CPU i pamięci.
\end{itemize}

\subsection{Opis komponentów aplikacji - 8 pkt}
\label{sec:FunctionalConditions}
\begin{itemize}
    \item Deployment metryki badajacy wykorzystanie zasobów (uruchomiony przed innymi komponentami).
    \item Klient: Zbudowana aplikacja frontendowa React Vite działająca na porcie 3000. Wdrażana jako Deployment z 3 replikami. Można się dostać do niej z zewnątrz dzięki Ingressowi Nginx.
    \item Serwer: Serwis backendowy ExpressJS działający na porcie 4000, który łączy się z bazą danych Neo4j. Dane potrzebne do połączenia się z bazą są przechowywane w Secret. Serwer wdrażany jest jako Deployment ze zmienną liczbą replik (od 3 do 6) dobieraną przez HorizontalPodScaler w zależności od trafficu. Loadbalancing jest zapewniony przez Ingress Nginx.
    \item Neo4j: Baza danych grafowa działająca na portach 7474 (HTTP) i 7687 (protokół Bolt). Jest wdrożona jako StatefulSet, co zapewnia trwałość i stabilność danych. Korzysta także z PersistentVolume i PersistentVolumeClaim, umożliwiających trwałe przechowywanie zebranych danych w lokalnej pamieci masowej.
\end{itemize}
Każdy komponent jest wdrażany jako osobny pod zarządzany przez Kubernetes, a Serwisy zapewniają stabilne identyfikatory sieciowe.

\subsection{Konfiguracja i zarządzanie - 4 pkt}
\label{sec:NonFunctionalConditions}
Konfiguracja aplikacji jest zarządzana za pomocą ConfigMaps i zmiennych \\środowiskowych przechowywanych w Secrets. Baza danych Neo4j jest inicjowana za pomocą skryptu przechowywanego w ConfigMap. Kubernetes zarządza wdrażaniem, skalowaniem i aktualizacją aplikacji, minimalizując przestoje. Aplikacje w klastrze sa konfigurowane poprzez pliki yaml. Korzystajac z kubectl w terminalu można usuwać i dodawać pody, zmieniać ich zasoby lub ustawienia oraz sprawdzać ich stan. Aktualizowanie aplikacji odbywa sie poprzez wprowadzenie odpowiednich zmian w pliku .yaml i jego ponowne zastosowanie (kubectl apply -f). Do cofniecia zmian służy komenda kubectl rollout undo.


\subsection{Zarządzanie błędami - 2 pkt}
\label{sec:ERD} 
Zarządzanie błędami obejmuje monitorowanie logów aplikacji, ustawianie livenessProbe i readinessProbe dla podów klienta i serwera. Zużycie zasobów poszczególnych kontenerów jest monitorowane przez Metrics Server, który z kolei przekazuje te dane do HorizontalAutoScaler. Automatyczne zdolności samonaprawcze Kubernetes automatycznie restartują nieudane kontenery.


\subsection{Skalowalność - 4 pkt}
\label{sec:ExamplesSection}
Skalowalność jest zapewniona dzięki użyciu Metrics Server w połączeniu z HorizontalAutoScaler. Kontener Metrics zbiera informacje o wykorzystaniu procesora i pamięci RAM bezpośrednio z kubeletów, co jest kluczowym elementem w procesie skalowania. Ten kontener współpracuje z HorizontalPodAutoscalerem, który umożliwia efektywne zarządzanie zasobami i redukcję ryzyka wystąpienia problemów dzięki regularnemu monitorowaniu zużycia mocy obliczeniowej procesora. Gdy wykryje on zbyt wysokie obciążenie, automatycznie tworzy kolejne repliki aplikacji zgodnie z określonymi w konfiguracji zasadami (od 3 do 6).
\subsection{Wymagania dotyczące zasobów - 2 pkt}
Odgórnie ustalone w plikach konfiguracyjnych wymagania dotyczące zasobów przydzielanych poszczególnym kontenerom:
\label{sec:ExampleResults}

\begin{table}[h!]
\centering
\begin{tabular}{|l|c|c|c|}
\hline
       & Klient & Serwer & Baza Danych \\ \hline
CPU    & 500m     & 1000m   & 2000m       \\ \hline
RAM    & 1500Mi   & 1024Mi  & 2048Mi      \\ \hline
\end{tabular}
\caption{Limity zasobów}
\label{tab:resource-limits}
\end{table}

Oczekiwania dotyczące wydajności obejmują niski czas odpowiedzi przy zapytaniach do bazy danych oraz wysoką dostępność. Faktyczne zużycie zasobów podczas ruchu na stronie uzyskane dzięki metrykom:
\begin{table}[h!]
\centering
\begin{tabular}{|c|c|c|c|}
\hline
 & \textbf{Klient} & \textbf{Serwer} & \textbf{Baza Danych} \\ \hline
\textbf{Zużycie RAM} & 800Mi & 700Mi & 500Mi \\ \hline
\textbf{Zużycie CPU} & 5m & 400m & 200m \\ \hline
\textbf{Miejsce na dysku} & 1.5Gi & 1.5Gi & 1Gi \\ \hline
\textbf{Czas odpowiedzi} & $<$ 2s & $<$ 400ms & $<$ 3s \\ \hline
\end{tabular}
\caption{Zużycie zasobów i czas odpowiedzi}
\label{table:zuzycie}
\end{table}


\subsection{Architektura sieciowa - 4 pkt}
Klastr Kubernetes został skonfigurowany z wykorzystaniem protokołu sieciowego CNI (Container Network Interface), który umożliwia każdemu kontenerowi w klastrze komunikację sieciową. W moim klastrze używam CNI pluginu, który zapewnia elastyczne zarządzanie adresami IP oraz konfigurację routingu między kontenerami.

\subsection*{Wykorzystywane Protokoły:}

\begin{enumerate}
    \item \textbf{HTTP}: Protokoły te są używane do komunikacji między klientem a serwerem e-commerce oraz do dostępu do aplikacji poprzez Ingress.
    
    \item \textbf{TCP}: Wykorzystywany głównie do komunikacji między kontenerami, na przykład pomiędzy frontendem (client) a backendem (server), oraz do komunikacji z bazą danych.
    
    \item \textbf{Bolt}: Protokół Bolt jest wykorzystywany do komunikacji między serwerem aplikacyjnym a bazą danych Neo4j, zapewniając szybki i efektywny dostęp do danych.
\end{enumerate}

\subsection*{Narzędzia do Zarządzania Siecią:}

\begin{enumerate}
    \item \textbf{Kubernetes Networking}: Zarządzanie siecią w klastrze Kubernetes obejmuje konfigurację Ingress, Service oraz Network Policies, które definiują reguły komunikacji między usługami oraz zarządzają dostępem do aplikacji.
    
    \item \textbf{NGINX Ingress Controller}: Kontroler Ingress NGINX zarządza ruchem sieciowym do aplikacji w klastrze, umożliwiając elastyczne routowanie ruchu HTTP do różnych usług w klastrze (klienta i serwera).
\end{enumerate}

\subsection{Podsumowanie}
\label{sec:ExampleResults}
Dzięki powyższej strukturze i wykorzystaniu potężnych możliwości orchestracji Kubernetes, aplikacja zapewnia wysoką dostępność, skalowalność oraz efektywne zarządzanie zasobami. Każdy komponent jest konteneryzowany i zarządzany niezależnie, co zapewnia elastyczność i odporność całego systemu.

\noindent

\bibliographystyle{amsplain}
\bibliography{references.bib}
\nocite{*}

\end{document}